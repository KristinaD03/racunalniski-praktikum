\documentclass{beamer}
\usepackage[utf8]{inputenc}
\usepackage[T1]{fontenc}
\usepackage[slovene]{babel}
\usepackage{lmodern}
\usepackage{url}
\usepackage{graphicx}

% 1. naloga: pripravite naslovno stran
% Nekaj o kompleksni dinamiki
% Beno Učakar
% Fakulteta za matematiko in fiziko
% 1. naloga: namig
% https://www.overleaf.com/learn/latex/Beamer#The_title_page

% 1. naloga: popravite preambulo tako, da bo imel temo "metropolis"
\usetheme{metropolis}
\usecolortheme{spruce}
\usefonttheme{structurebold}
\useoutertheme{infolines}
\beamertemplatenavigationsymbolsempty

% 2. naloga: definirajte novo AMS okolje "definicija"
\usepackage{amsmath}
\newtheorem{definicija}{Definicija}

\begin{document}

\title{Nekaj o kompleksni dinamiki}
\author{Beno Učakar}
\institute{Fakulteta za matematiko in fiziko}
\date{\today}

\begin{frame}
    \titlepage
\end{frame}

% 1. naloga: pripravite naslovno stran

\begin{frame}
 \onslide<1->
    Kompleksna števila lahko enostavno vstavimo v polinom, kaj pa druge funkcije? 
    Za primer si poglejmo, kako izračunamo $e^{i\theta}$ s pomočjo Taylorjeve vrste.
    
    % 3. naloga: okolje za poravnano enačbo
    % začetek okolja
\onslide<2->
    \begin{align*}
         e^{i\theta} = \sum_{n=1}^{\infty} \frac{(i\theta)^n}{n!}
                     &= 1 + i\theta - \frac{\theta^2}{2!} - \frac{i\theta^3}{3!} + \frac{\theta^4}{4!} + \frac{i\theta^5}{5!} + \ldots=\\
                     &= \left( 1 - \frac{\theta^2}{2!} + \frac{\theta^4}{4!} - \ldots \right) 
                        + i \left(\theta - \frac{\theta^3}{3!} + \frac{\theta^5}{5!} - \ldots \right)=\\
                     &= \cos\theta + i\sin\theta.
    \end{align*}
    % konec okolja
\end{frame}

\begin{frame}
    % 2. naloga: uporabite okolje definicija in poudarite izraz "večkratnost"
    % začetek definicije
    \begin{definicija}
        Naj bo $f \in O(D)$ in $z_0 \in D$ fiksna točka funkcije $f$.
        Število $\lambda = f'(z_0)$ imenujemo \textbf{večkratnost} funkcije $f$ v točki $z_0$.
    \end{definicija}
    % konec definicije
\end{frame}

    % 4. naloga: prelom prosojnice
\begin{frame}
    \begin{exampleblock}{Glede na $\lambda$ karakteriziramo fiksne točke:}
        % 4. naloga: postopno prikazovanje elementov in manjkajoč izraz
        \begin{enumerate} 
            \item<+-> $|\lambda| = 0$ je \textbf{super privlačna} fiksna točka.
            \item<+-> $|\lambda| < 1$ je \textbf{privlačna} fiksna točka.
            \item<+-> $|\lambda| > 1$ je \textbf{odbojna} fiksna točka.
            \item<+-> $|\lambda| = 1$: 
                če je $\lambda^n \neq 1$ za vsak \(n\in \mathbb{N}\) je fiksna točka \textbf{iracionalno},
                sicer pa \textbf{racionalno nevtralna}.
        \end{enumerate}
    \end{exampleblock}
\end{frame}

\begin{frame}
    % 5. naloga: naslov prosojnice
    % Primer Julijeve množice
    \frametitle{Primer Julijeve množice}
    Julijeva množica za funkcijo $f(z) = z^2 + c$ je množica vseh točk v kompleksni ravnini, katerih 
    orbite pod iteracijami funkcije $f$ ostajajo znotraj neke omejene regije.
    \newline\newline
    Za različne vrednosti kompleksnega parametra $c$ dobimo različne Julijeve množice.
    \newline\newline        
    \includegraphics[width=0.5\textwidth]{julia_set.png}

    % 5. naloga: vstavite sliko julia_set.png (potrebujete tudi ustrezen paket)
\end{frame}

\end{document}