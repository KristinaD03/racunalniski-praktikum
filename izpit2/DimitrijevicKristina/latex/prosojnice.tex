\documentclass{beamer}
\usepackage[slovene]{babel}
\usepackage[utf8]{inputenc}
\usepackage[T1]{fontenc}
\usepackage{lmodern}
\usepackage{amsmath, amssymb, bbm}
\usepackage{graphicx}
\usepackage{url}
\usepackage{citeall}


\usetheme{metropolis}
\beamertemplatenavigationsymbolsempty
\setbeamertemplate{caption}[numbered]

\usepackage{palatino}
\usefonttheme{serif}
{\theoremstyle{definition}
\newtheorem{definicija}{Definicija}}

% Krožne matrike
% Samo Primer


% 
% Pozor: dokler ne dodate vsaj enega okolja za prosojnico, 
% se datoteka ne bo prevedla.
% 
\title{Krožne matrike}
\begin{document}
\author{Samo Primer}
\institute[FMF]{Fakulteta za matematiko in fiziko}
\date{}
\maketitle

\begin{frame}{definicija}
 \begin{definicija}
        Krožna matrika  \(n\times n\) je matrika oblike
 \end{definicija}
        \[\begin{pmatrix}
         c_0 & c_{n-1} &  c_2 & c_1\\
         c_1 & c_0  &  c_{n-1} & c_2\\
         c_2 & c_1  &  c_0 & c_{n-1}\\
         \vdots & \vdots  & \ddots & \vdots \\ 
         c_{n-1}&  c_{n-2}&    c_1&  c_0\\
        \end{pmatrix}
        \]
       
\end{frame}

% začetek definicije
 

        % c_0  c_{n-1}    c_2  c_1 
        % c_1  c_0    c_{n-1}  c_2 
        % c_2  c_1    c_0  c_{n-1} 
        % 
        % c_{n-1}  c_{n-2}    c_1  c_0
        
% konec definicije



% začetek neoštevilčenega seznama
\begin{frame}{lastnosti}
        \begin{itemize}
        % lastnost 1
        \item
        \onslide<1->Lastni vektorji krožne matrike so 
        $v_j = (1, \omega_j, \omega_j^2, \ldots, \omega_j^{n-1})^T$, 
        kjer je \(omega_j= e^{\frac{2\pi ij}{n}}\) za $j=0, \ldots, n-1$.
        \item 
        \onslide<2->Lastni vektorji krožne matrike so stolpci matrike enotske diskretne Fourierjeve transformacije, 
        ki jo lahko prikažemo kot Vandermondovo matriko.
        % lastnost 3
        \item
        \onslide<3->Krožne matrike tvorijo komutativno algebro, ker je za poljubni dve matriki 
        $A$ in $B$, vsota $A + B$ tudi krožna matrika, prav tako je krožna matrika tudi njun produkt 
        $AB$, ter tudi velja $AB = BA$.
        \end{itemize}
\end{frame}
% konec neoštevilčenega seznama
\begin{frame}{literatura}
\bibliographystyle{plain}
\bibliography{literatura}
\cite{gray}
\end{frame}



\end{document}

